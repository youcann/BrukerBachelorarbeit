\chapter{Einleitung}

\gls{mrt} ist nicht nur in der Humanmedizin, sondern auch in der Kleintierbildgebung eine wichtige Modalität.

 Hinsichtlich der Abbildung von Gewebe ist es anderen bildgebenden Verfahren, wie der \gls{ct}, überlegen: Statt mit Röntgenstrahlung, wie beim \gls{ct}, entstehen die Schnittbilder beim \gls{mrt}-Tomographen durch Ausnutzung des \gls{nmr}-Effekts der Wasserstoffkerne im Körper, welcher einen hohen Weichteilkontrast ermöglicht. Eine \gls{mrt}-Untersuchung verursacht keine Belastung mit ionisierenden Strahlen.
 
\section{Aufgabenstellung}
\tikzstyle{every node}=[yshift=4pt]
\input{img/trx1200.tex}

\todoin[]{Soll/Ist und Anforderungen an die Simulation}