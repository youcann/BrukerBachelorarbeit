\chapter{Material und Methoden}

\section{MRiLab}

MRiLab wird auf einem PC mit einer High-End Consumer Grafikkarte installiert (siehe \autoref{sec:usedPC}). 

\subsection{Analyse Bildrekonstruktionsalgorithmus}

\section{Bestimmung notwendiger Simulationsparameter}

\subsection{Berechnung Empfängerbandbreite}

Die Lamorfrequenzen an den Rändern des Probenraums sind:
\begin{subequations}
	\begin{align}
	f_{min}=\frac{\gamma}{2\pi} (B_0+G x_{min}) \\
	f_{max}=\frac{\gamma}{2\pi} (B_0+G x_{max})
	\end{align}
\end{subequations}
mit:
\begin{with}
	\frac{\gamma}{2\pi} =\SI{42.577478}{\mega\hertz\per\tesla}& Gyromagnetisches Verhältnis für \ce{^{1}H}-Kerne \\
	B_0 \text{~(in \SI{}{\tesla})} & Feldstärke statisches Magnetfeld \\
	G \text{~(in \SI{}{\tesla\per\meter})} & Gradientenstärke \\
\end{with}

Die notwendige Empfängerbandbreite $\Delta f$ ergibt sich daraus zu:
\begin{equation}
\Delta f = f_{max}-f_{min} = \frac{\gamma}{2\pi} G (x_{max}-x_{min})
\end{equation}
Da die Länge $x_{max}-x_{min}$ der Größe des $FOV$ in der betrachteten Richtung entspricht, kann $\Delta f$ berechnet werden mit:
\begin{equation}
\label{eq:deltaF}
\Delta f(G, FOV) = G \cdot FOV
\end{equation}

Mit Hilfe von \autoref{eq:deltaF} und typischen Werten für die Gradientenstärke und das FOV werden einige Empfängerbandbreiten berechnet (\autoref{tab:empfBandbreite}).

\begin{table}[H]
	\centering
	\caption{Notwendige Empfängerbandbreiten in Abhängigkeit des Field of View (FOV) und der Gradientenstärke}
	\label{tab:empfBandbreite}
	\begin{tabular}{llll}
		\toprule
		& \multicolumn{3}{c}{\textbf{G (in \SI{}{\milli\tesla\per\meter})}} \\ \cmidrule{2-4}
		\textbf{FOV (in \SI{}{\mm})}& 200 & 500 & 1000 \\
		20 & \SI{0.170}{\mega\hertz} & \SI{0.425}{\mega\hertz} & \SI{0.851}{\mega\hertz} \\
		50 & \SI{0.426}{\mega\hertz} & \SI{1.06}{\mega\hertz} & \SI{2.13}{\mega\hertz} \\
		150 & \SI{1.277}{\mega\hertz} & \SI{3.19}{\mega\hertz} & \SI{6.39}{\mega\hertz} \\
		\bottomrule
	\end{tabular}
\end{table}

\section{Generierung von farbigem Phasenrauschen}

\subsection{Grundsätzliche Vorgehensweise}

\subsection{Interpolation im Frequenzbereich}

\subsection{FIR Filter}

\section{Phasenrauschfunktion für MRiLab}

\section{Metriken zur Bildqualitätsbewertung}
