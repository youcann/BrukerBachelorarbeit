\chapter*{Abstract}
\thispagestyle{empty}
Magnetic resonance imaging (MRI) is used to measure cross-sectional images in a living creature or to get answers in research studies.

Increasing technical and economical requirements push the development of new electronic sub systems in the signal path of a MRI. The technological relationship between MRIs and NMR (nuclear magnetic resonance) spectrometers, which both use the NMR effect, was used in the past by Bruker BioSpin to share common parts among them.
Due to the higher stability requirements of the (reference) oscillators in NMR, this had no negative effect on MRI but in an economical sense, it is not optimal.

To optimize the signal path electronics and to develop a MRI exclusive acquisition electronic, it is necessary to assess influences on the quality of the reconstructed MRI image. A computer simulation can be used to get results fast and cost effective. By how much the image quality is allowed to deteriorate is currently only evaluated visually by the MRI application engineers.

In this work, \textit{MRiLab}, a MRI simulation framework based on MATLAB, is assessed. First the required simulation parameters are fitted to the needs of the Bruker product line up. Then a digital representation of an existing MRI phantom (a liquid-filled test body made from plastic) is created. It allows to verify the simulation results in the future using a real MRI machine.
Simulating some phase noise stochastic processes typical for real oscillators shows how different phase noise spectra affect image quality. To assess image quality degradation, common full reference image quality metrics, like MSE (mean squared error) or SSIM (structural similarity), are studied. Additionally some non-reference quality metrics, like NIQE (Naturalness Image Quality Evaluator) are evaluated for comparison.

The work shows that MRiLab is a capable tool for qualitative MRI simulations and can be used to aid the ongoing development process for new acquisition electronics. However, due to minor bugs and inaccuracies in the program, results have to be considered with care. Since different image errors and artifacts are not evaluated separately, existing image quality metrics are not fully suited for MRI quality assessment.

\chapter*{Kurzfassung}
\thispagestyle{empty}
Mit Magnetresonanztomographen (MRT) werden Schnittbilder von lebendigen Körpern erzeugt, um eine Diagnose zu stellen, oder Fragestellungen aus der Forschung zu beantworten.

Gestiegene technische und ökonomische Anforderungen an kommerzielle Magnetresonanz-Abbildungssysteme treiben die Entwicklung neuer elektronischer Baugruppen im Signalpfad der Geräte voran. Durch die technologische Verwandtschaft von MRTs und NMR (Nuclear Magnetic Resonance)-Spektrometern, die beide den Effekt der Kernspinresonanz ausnutzen, war es in der Vergangenheit bei Bruker BioSpin naheliegend, Entwicklungen aus dem Bereich der NMR auch in der MRT einzusetzen. Durch die höheren Stabilitätsanforderungen an die (Referenz-) Oszillatoren in der NMR hat dies zu keinerlei technischen Schwierigkeiten in der MRT geführt, ist jedoch unter der Berücksichtigung wirtschaftlicher Gesichtspunkte nicht optimal.

Für eine Optimierung der Signalkette und die Entwicklung einer MRT exklusiven Elektronik ist es notwendig, die Auswirkungen auf die Bildqualität der rekonstruierten MRT-Bilder abschätzen zu können. Dies ist mit Computersimulationen kostengünstig und schnell umsetzbar. In wie weit sich die Bildqualität reduzieren darf und welche Bildartefakte besonders störend sind, wird zur Zeit lediglich rein visuell von den Applikationsingenieuren der MRT-Systeme bewertet.

In dieser Arbeit wurde \textit{MRiLab}, eine MATLAB-basierte, Simulationsumgebung für die Magnetresonanzbildgebung näher untersucht. Dabei wurden zunächst die verwendeten Simulationsparameter auf die Bruker Produktpalette angepasst. Durch die digitale Nachbildung eines bereits existierenden MRT-Phantoms (einem flüssigkeitsgefüllten Prüfkörper aus Kunststoff) wurde ein Eingangsdatensatz erstellt, mit dem sich die Simulation auch mit Messungen realer MRTs verifizieren lässt. Durch die Simulation einiger, für Oszillatoren typischen, Phasenrauschprozesse konnte der Einfluss von Phasenrauschen auf die MR-Empfangssignale nachgebildet werden. Damit konnte gezeigt werden, wie sich verschiedene spektrale Leistungsdichten der Rauschprozesse auf die rekonstruierten Schnittbilder auswirken.
Um die dadurch hervorgerufene Bilddegradierung quantitativ bewerten zu können, wurden einige verbreitete, referenzbasierte (MSE (Mean Squared Error), SSIM (Structural SIMilarity)) und eine Auswahl an neuartigen, referenzfreien (z.B. NIQE (Naturalness Image Quality Evaluator)) Bildqualitätsmetriken untersucht.

Es konnte gezeigt werden, dass sich MRiLab als nützliches Werkzeug für qualitative Simulationen im weiteren Entwicklungsprozess der MRT-Signalkette eignet. Durch einige kleinere Fehler und Ungenauigkeiten im Programm müssen die Ergebnisse jedoch kritisch bewertet werden. Da unterschiedliche Bildfehler und Artefakte nicht differenziert und separat gewichtet werden, eignen sich typische Bildqualitätsmetriken, wie SSIM, nur bedingt für die MRT.


