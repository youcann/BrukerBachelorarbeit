\chapter{Grundlagen}
In diesem Kapitel werden einige Grundlagen erläutert, die für den weiteren Verlauf der Arbeit relevant sind.

\section{(Präklinische-) Magnetresonanztomographie}
Magnetresonanztomographie (abgekürzt MRT\footnote{Auch \textit{MRI} (englisch magnetic resonance imaging)}) bezeichnet ein bildgebendes Verfahren, dass den physikalischen Effekt der Kernspinresonanz (NMR nach englisch nuclear magnetic resonance) nutzt.

Als \textit{Tomografie}\footnote{von altgriechisch tome, Schnitt und graphein, schreiben}-Verfahren ermöglicht es MRT, dreidimensionale Strukturen als Schnittbilder aufzunehmen und wieder in 3D darzustellen. Die Schnittbilder geben die innere Struktur eines Objektes dabei so wieder, als wäre das Objekt tatsächlich in der betrachteten Ebene geschnitten. Insbesondere sind die Bilder im Vergleich zu Projektionsverfahren (wie z.B. der klassischen Röntgenaufnahme) überlagerungsfrei.

Im Gegensatz zur \textit{Computertomographie} (CT), einem weiteren sehr verbreitetem Tomographieverfahren, kommt bei einer MRT Untersuchung keine ionisierende Strahlung zum Einsatz.

Die folgenden Abschnitte beschreiben den NMR-Effekt und die weiteren Techniken, die zur Bildentstehung in einem MRT Gerät nötig sind. Im letzten Abschnitt wird auf die Besonderheiten der präklinischen Bildgebung eingegangen.

\subsection{Der NMR-Effekt}

\subsection{Aufbau eines MR-Tomographen und Bildentstehung}

\subsection{MRT Pulssequenzen}

\subsection{Besonderheiten präklinischer MRT Systeme}

\subsection{Vergleich mit anderen Modalitäten}




\section{Rauschen}
In der Elektronik bezeichnet \textit{Rauschen} eine, zumeist unerwünschte, nicht-deterministische Störung eines elektrischen Signals.

\subsection{Phasenrauschen}