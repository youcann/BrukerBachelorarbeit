\chapter{Grundlagen}
In diesem Kapitel werden einige Grundlagen erläutert, die für den weiteren Verlauf der Arbeit relevant sind.

\section{(Präklinische-) Magnetresonanztomographie}
Magnetresonanztomographie (abgekürzt MRT\footnote{Auch \textit{MRI} (englisch magnetic resonance imaging)}) bezeichnet ein bildgebendes Verfahren, dass den physikalischen Effekt der Kernspinresonanz (NMR nach englisch nuclear magnetic resonance) nutzt.

Als \textit{Tomografie}\footnote{von altgriechisch tome, Schnitt und graphein, schreiben}-Verfahren ermöglicht es MRT, dreidimensionale Strukturen als Schnittbilder aufzunehmen und wieder in 3D darzustellen. Die Schnittbilder geben die innere Struktur eines Objektes dabei so wieder, als wäre das Objekt tatsächlich in der betrachteten Ebene geschnitten. Insbesondere sind die Bilder im Vergleich zu Projektionsverfahren (wie z.B. der klassischen Röntgenaufnahme) überlagerungsfrei.

Im Gegensatz zur \textit{Computertomographie} (CT), einem weiteren sehr verbreitetem Tomographieverfahren, kommt bei einer MRT Untersuchung keine ionisierende Strahlung zum Einsatz.

Die folgenden Abschnitte beschreiben den NMR-Effekt und die weiteren Techniken, die zur Bildentstehung in einem MRT Gerät nötig sind. Im letzten Abschnitt wird auf die Besonderheiten der präklinischen Bildgebung eingegangen.

\subsection{Der NMR-Effekt}
Die Entwicklung der Magnetresonanzbildgebung wurde erst durch die Entdeckung der Kernspinresonanz möglich. Der physikalische Effekt wird auch außerhalb englischsprachiger Länder meist \textit{nuclear magnetic resonance}, kurz NMR, genannt.

\subsubsection{Geschichte}
1922 zeigten \textsc{Stern} und \textsc{Gerlach} im, nach ihnen benannten, Stern-Gerlach-Experiment die Quantelung von (Eigen-) Drehimpulsen: Silberatome werden aus einem Ofen heraus durch ein inhomogenes Magnetfeld auf eine Glasplatte beschleunigt. Klassisch wäre durch die Ablenkung des Magnetfeldes eine kontinuierliche Intensitäts-Verteilung um $0\degree$ auf der Platte zu erwarten. Stattdessen entstehen durch die quantisierten Spins, zwei diskrete Punkte. \cite{Gerlach1922}

1938 konnte die Kernspinresonanz erstmals von \textsc{Isidor Isaac Rabi} durch eine Erweiterung des Stern-Gerlach-Versuchs nachgewiesen werden: Dazu wird die Anordnung des Stern-Gerlach-Versuchs um ein weiteres inhomogenes Magnetfeld dahinter ergänzt. Dieses ist umgekehrt gepolt und ansonsten identisch dem ersten Feld. Das erste Feld wirkt daher de-fokussierend auf die Atome, das zweite re-fokussierend. Wird ein homogenes Wechsel-Magnetfeld zwischen den beiden inhomogenen Feldern erzeugt, sinkt die gemessene Detektorintensität hinter der Anordnung bei einer bestimmten Frequenz. Bei dieser Frequenz $\omega_0$ wird die Resonanzbedingungen zwischen zwei Übergängen erreicht. $\omega_0$ heißt daher Resonanzfrequenz. Damit ist es möglich, die benötigte Energie zu bestimmen, um ein atomares, magnetisches Moment anzuregen. Für seine Arbeiten erhielt Rabi 1944 den Nobelpreis für Physik. \cite{Rabi1938}

Weitere Pionierarbeit auf dem Gebiet der NMR leisteten der Schweizer Physiker \textsc{Felix Bloch} und der US-Amerikaner \textsc{Edward Mills Purcell}.

\subsubsection{Quantenmechanische Erklärung}




\subsection{Aufbau eines MR-Tomographen und Bildentstehung}
Bild, Magnet, Kühlung, Gradienten

\subsection{MRT Pulssequenzen}
\begin{enumerate}
	\item Spin-Echo
	\item Gradienten-Echo
	\item EPI
\end{enumerate}


\subsection{MRT Artefakte}
Zahlreiche Effekte können zu Artefakten in MRT Aufnahmen führen. Welche Artefakte dominieren und ob die Bildqualität wesentlich verschlechtert wird, hängt stark von der Anwendung ab.

Im Wesentliche können die Artefakt-Ursachen in drei Kategorien eingeteilt werden: Gewebeartefakte, Bewegungsartefakte und Artefakte durch die Technik des Tomografen.

\subsubsection{Gewebeartefakte}

\subsubsection{Bewegungsartefakte}

\subsubsection{Technische Artefakte}

\subsection{Besonderheiten präklinischer MRT Systeme}
Größe, Handhabungssysteme

\subsection{Vergleich mit anderen Modalitäten}




\section{Rauschen}
In der Elektronik bezeichnet \textit{Rauschen} eine, zumeist unerwünschte, nicht-\-deterministische Störung eines elektrischen Signals.


\subsection{Phasenrauschen}