\chapter{Zusammenfassung und Ausblick}
In dieser Arbeit wurde die Eignung von MRiLab, einer auf MATLAB basierenden Software, für die Simulation von präklinischen MR-Tomographen untersucht.

Um bekannte Eingangsdaten für die Simulation zu erhalten, die auch in Zukunft Gültigkeit behalten und durch reale Messungen verifizierbar sind, wurde ein bereits vorhandenes MRT-Phantom als \gls{cad}-Modell nachgebildet und so konvertiert, dass es für die Simulation nutzbar ist.

Mit diesem Phantom wurden anschließend MRT-Simulationen mit verschiedenen Pulssequenzen durchgeführt. Damit konnte die grundlegende Funktion von MRiLab gezeigt werden. Durch Hinzufügen von Phasen\-rauschen mit unterschiedlicher spektraler Leistungsdichte auf das MR-\-Empfangs\-signal vor bzw. nach Eintrag in den k-Raum, konnte der Einfluss dieses Phasenrauschens auf die Bildqualität rein visuell beobachtet werden. Weißes Phasenrauschen ist dabei für Spinecho- und EPI-Sequenzen unproblematisch. Phasenrauschen mit einem $1/f$-Anteil hingegen beeinträchtigt die Bildqualität unter Verwendung einer EPI-Sequenz stark. Auch andere Sequenzen mit langen k-Raum Trajektorien zeigen eine ähnliche Empfindlichkeit auf niederfrequentes Phasenrauschen. Verschiedene Phasenrauschspektren eines \textit{LMK0482x} (Jittercleaner IC) in der Nähe eines MRT und in einem MRT wurden aufgenommen. Mit den Aufnahmen zeigte sich, dass der LMK0482x in einigen Zentimetern Abstand zum Feld der Gradientenspulen nur wenig zusätzliches Phasenrauschen erzeugt. Die Bildqualität wird dadurch nicht beeinträchtigt. Wird der IC direkt in den Gradientenspulen betrieben, steigt das Phasenrauschen stark an und die Bildqualität wird sehr schlecht.

Zur quantitativen Bewertung der Ergebnisse wurden Bildqualitätsmetriken, wie die \gls{ssim} oder der \gls{mse} verwendet. Diese eignen sich nur bedingt für die Bildqualitätsbewertung in der MRT, da stark störende Artefakte, wie Geisterbilder nicht stärker gewichtet werden, als weniger gravierende Bildfehler, wie z.B. Bildrauschen durch ein geringes SNR.

Hilfreich für die weitere Optimierung der Signalkette wäre daher eine Bildqualitätsmetrik, die auf die Anforderungen der \gls{mrt} zugeschnitten ist und deren Modell trainierbar ist. Mit Hilfe von Referenzbildern aus der Anwendungsentwicklung sollte ein Maß entstehen, das unterschiedliche Arten von Bilddegradationen danach gewichtet, wie sie sich auf die Aussagekraft des MRT-Bildes auswirken. So ist ein SNR-Verlust meist eher tolerierbar, als das Auftreten von Geisterbildern, die anatomische Strukturen vortäuschen können. 