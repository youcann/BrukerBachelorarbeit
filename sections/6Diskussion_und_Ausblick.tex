\chapter{Zusammenfassung und Ausblick}
In dieser Arbeit wurde die Eignung von MRiLab, einer auf MATLAB basierenden Software, für die Simulation von präklinischen MR-Tomographen untersucht.

Um bekannte Eingangsdaten für die Simulation zu erhalten, die auch in Zukunft Gültigkeit behalten und durch reale Messungen verifizierbar sind, wurde ein bereits vorhandenes MRT-Phantom als \gls{cad}-Modell nachgebildet und so konvertiert, dass es für die Simulationen nutzbar ist.

Mit diesem Phantom wurden anschließend MRT-Simulationen mit verschiedenen Pulssequenzen durchgeführt. Damit konnte die grundlegende Funktion von MRiLab gezeigt werden. Durch Hinzufügen von Phasen\-rauschen mit unterschiedlicher spektraler Leistungsdichte auf das MR-\-Empfangs\-signal vor Eintrag in den k-Raum, konnte der Einfluss dieses Phasenrauschens auf die Bildqualität rein visuell beobachtet werden.

Dabei zeigte sich, dass sich Phasenrauschen mit einem $1/f$-Anteil besonders negativ auf die Bildqualität von EPI-Schnittbildern auswirkt. Auch andere Sequenzen mit langen k-Raum Trajektorien zeigen eine ähnliche Empfindlichkeit auf niederfrequentes Phasenrauschen. Für realitätsnahe Abschätzungen wurde das Phasenrauschen des Jittercleaner-Bausteins LMK04821 betrachtet. Es zeigte sich, dass das Phasenrauschen durch ein starkes statisches Magnetfeld nicht verändert wird. Starke modulierte Magnetfelder, wie die Gradientenfelder eines MRT, führen jedoch bei dem Baustein zu einem starken Phasenrauschen bzw. einer Phasenmodulation. Daher wurden Simulationen mit dem Phasenrauschleistungsspektrum des LMK04821 und zusätzlich mit reinem $(1/f)$-Rauschen bzw. einer reinen Sinus-Phasenmodulation durchgeführt.

Zur quantitativen Bewertung der Ergebnisse wurden die Bildqualitätsmetriken \gls{ssim} und \gls{mse} verwendet. Diese eignen sich nur bedingt für die Bildqualitätsbewertung in der MRT, da stark störende Artefakte, wie Geisterbilder, nicht stärker gewichtet werden, als weniger gravierende Bildfehler, wie z.B. Bildrauschen durch ein geringes SNR. Die Metriken können jedoch nützlich sein, um Bilder mit unterschiedlichen Aufnahmeparametern zu vergleichen.

Hilfreich für die weitere Optimierung der Signalkette wäre daher eine Bildqualitätsmetrik, die auf die Anforderungen der \gls{mrt} zugeschnitten ist und mit der eine absolute Bewertung möglich ist. Mit Hilfe von Referenzbildern aus der Anwendungsentwicklung sollte ein Maß entstehen, das unterschiedliche Arten von Bilddegradationen danach gewichtet, wie sie sich auf die Aussagekraft des MRT-Bildes auswirken. So ist ein SNR-Verlust meist eher tolerierbar als das Auftreten von Geisterbildern, die anatomische Strukturen vortäuschen können. 